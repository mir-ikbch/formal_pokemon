\documentclass{jsarticle}
\usepackage{amsmath,amsfonts,amsthm}
\theoremstyle{definition}
\newtheorem{definition}{定義}
\newtheorem{example}{例}

\begin{document}

ポケモンバトルの概要:二人のプレイヤーA,Bがそれぞれ $n$ 体のポケモンを用いて勝負する.
各ポケモンは体力,攻撃力,防御力,素早さのパラメータ,いくつかの技と,タイプと呼ばれる情報を持つ.
技には相手のポケモンの体力を減らすもの(攻撃技)や自分の体力を回復するもの(回復技)がある.
バトル中には各プレイヤーは常に1体のポケモンを場に出しており,他のポケモンは控えに置く.

各ターン,プレイヤーは(1)場に出ている自分のポケモンの技を使う,(2)場の自分のポケモンを,自分の控えにいる体力が0より大きいポケモン1体と交換する,のどちらかを選ぶ.
各プレイヤーは自分の行動を同時に発表する.双方の場のポケモンが技を使う場合,素早さの速いポケモンの技が先に発動する.
どちらかのプレイヤーが控えとの交換を選び,他方が技を選んだ場合,交換は技よりも先に起こる.
技によってポケモンの体力が0になった場合,その所持者であるプレイヤーは控えから体力が0でないポケモンを選び,交換する.
技を使って相手のすべてのポケモンの体力を0にしたプレイヤーが勝者となる.

攻撃技には威力 $d$ と命中率 $p$ というパラメータとタイプを持つ.
技は $1-p$ の確率で失敗し,失敗した場合技の効果は発動しない.
成功した場合,場にいる相手のポケモンの体力を減らす.
減らす量(ダメージ)は $\lfloor(\alpha da/b + \beta)e(t_1,t_2)\rfloor$ である.
ここで,$a$ は技を使う側のポケモンの攻撃力,$b$ は他方の場のポケモンの防御力 $b$,$t_1$ は技のタイプ,$t_2$ は相手の場のポケモンのタイプ,$e$ はバトルの前に固定された関数,$\alpha,\beta$ はバトルの前に固定された正定数である.

回復技は回復率 $r\in[0,1]$ というパラメータを持つ.回復技を使うとそのポケモンは自身の最大体力の $r$ 倍の体力を回復する.

\section{形式化}
正整数 $n$,正実数 $\alpha,\beta$,集合 $T$,$e\colon T \times T \to \mathbb{R}_{\ge 0}$ を固定する.
$n$ はバトルにおいて各プレイヤーが何体のポケモンを持つかを表し,$\alpha,\beta$ はダメージの計算に使う定数である.
集合 $T$ の要素はタイプと呼ばれ,$e$ はタイプ相性関数と呼ばれる.
\begin{definition}
(技)攻撃技とは,威力と呼ばれる正整数 $d$ と命中率と呼ばれる実数 $p\in [0,1]$,タイプ $t\in T$ の3つ組から成る.
回復技とは,回復率と呼ばれる実数 $r \in [0,1]$ から成る.
攻撃技と回復技を合わせて技と呼び,すべての技の集合を $\mathcal M$ と書く.
\end{definition}
\begin{definition}
(ポケモン)ポケモンとは,正整数 $H,A,B,S$,タイプ $t \in T$,技の集合 $M\subset \mathcal M$ の6つ組である.
ここで,$H$ は最大体力もしくは最大ヒットポイント(最大HP),$A$ は攻撃力,$B$ は防御力,$S$ は素早さと呼ばれる.
\end{definition}
二人のプレイヤーを $\mathcal A,\mathcal B$ とする.
バトルは,プレイヤー $\mathcal A$ には $n$ 体のポケモン $P_{\mathcal A,1},\dots,P_{\mathcal A,n}$,$\mathcal B$ には $P_{\mathcal B,1},\dots,P_{\mathcal B,n}$ が割り当てられている状態から始まる.
ポケモン $P_{X,i}$ に対する上記のパラメータ $H,A,\dots$ をそれぞれ $H_{X,i},A_{X,i},\dots$ などと書く.
\begin{definition}
(局面)ゲーム中の局面は,($2n+2$)-組 $(i_A,i_B,h_{\mathcal A,1},\dots,h_{\mathcal A,n},h_{\mathcal B,1},\dots,h_{\mathcal B,n})$ で表される.
ここで,$i_\mathcal A,i_\mathcal B$ はそれぞれ $\mathcal A,\mathcal B$ の場にいるポケモンが $P_{\mathcal A,i_\mathcal A},P_{\mathcal B,i_\mathcal B}$ であることを表し,$h_{X,i}$ はポケモン $P_{X,i}$ の体力もしくはヒットポイント(HP)と呼ばれる.
通常,$h_{X,i}$ が区間 $[0,H_{x,i}]$ 上にあるようなものだけ扱う.

局面 $(1,1,H_{\mathcal A,1},\dots,H_{\mathcal A,n},H_{\mathcal B,1},\dots,H_{\mathcal B,n})$ を初期局面と呼び,バトルはこの局面から始まる.
\end{definition}
\begin{definition}
(局面遷移)局面を遷移させるために,プレイヤー $X$ は以下のいずれかの行動を取る:
\begin{enumerate}
	\item (技を繰り出す)場のポケモン $P_{X,i_X}$ の技 $m\in M_{X,i_X}$ を一つ選ぶ.
	\item (場のポケモンを交換する)場にいない自分のポケモン $P\in \{P_{X,1},\dots,P_{X,n}\}\setminus\{P_{X,i_X}\}$ を一体選ぶ.
\end{enumerate}
プレイヤー $\mathcal A,\mathcal B$ は行動を決めた後同時にそれを発表する.
現在の局面 $(i_\mathcal A,i_\mathcal B,h_{\mathcal A,1},\dots,h_{\mathcal A,n},h_{\mathcal B,1},\dots,h_{\mathcal B,n})$ は以下の手順で(破壊的)更新する:
\begin{itemize}
	\item プレイヤー $X$ がポケモン $P_{X,i}$ への交換を選んだ場合,$i_X := i$ とする.
	\item 技 $m$ を選んだプレイヤー $X$ について以下を行う.双方のプレイヤーが技を選んだ場合,素早さ $S_{X,i_X}$ がより大きい方の処理を先に行う.
素早さが同じ場合,1/2の乱数で先行を決定する.
先に行われた技によって,後に行動するポケモンの体力 $h_{X,i_X}$ が0になった場合,そのポケモンの技の処理は行われない.
	\begin{itemize}
		\item $m$ が攻撃技 $(d,p,t)$ の場合:区間 $[0,1]$ 上の一様乱数を生成し,それが $p$ 以下の場合,$h_{\bar X,i_{\bar X}} := \min(0,h_{\bar X,i_{\bar X}} - \lfloor(\alpha dA_{X,i_X}/B_{\bar X,i_{\bar X}} + \beta)e(t,t_{\bar X,i_{\bar X}}) \rfloor)$ と更新する.
ここで,$\bar X$ は $X=\mathcal A$ のとき $\mathcal B$,$X=\mathcal B$ のとき $\mathcal A$ を表す.
		\item $m$ が回復技の場合:$m$ の回復率が $r$ のとき,$h_{X,i_X} := \max(H_{X,i_X},\lfloor h_{X,i_X} + rH_{X,i_X}\rfloor)$ と更新する.
	\end{itemize}
	\item 場のいずれかのポケモンの体力が0になった場合,その所持者は控えから体力が0でない自身のポケモン $P_{X,i}$ を選び,$i_X := i$ と更新する.
自身のすべてのポケモンの体力が0の場合,そのプレイヤーは敗者となり,バトルは終了する.
\end{itemize}
\end{definition}

\end{document}